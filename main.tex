%%=======================================================================
% !Mode:: "TeX:UTF-8"
% !TEX program  = PdfLaTeX
%%=======================================================================
% 模板名称:thubeamer
% 模板版本:V1.1.1
% 模板作者:杨敬轩(Jingxuan Yang)
% 联系作者:yangjx20@mails.tsinghua.edu.cn & yanglatex2e@gmail.com
% 模板适用:清华大学风格 Beamer 模板
% 模板编译:手动编译方法参看 README.md 或 thubeamer.pdf
%          编译 beamer 之前必须编译说明文档:make doc 或双击 makedoc.bat
%          编译说明文档同时分离出四个样式文件 *thubeamer.sty
%          GNU make 工具:make beamer
%          Windows 批处理脚本:双击 makebeamer.bat 自动编译 beamer
%          更多编译细节详见说明文档:thubeamer.pdf
% 更新时间:2022/05/20
% 模板帮助:请**务必务必务必**阅读 thubeamer.pdf 说明文档,文档查看方法:
%          下载模板文件夹里就有,如果是从 CTAN 上安装更新本模板,则通过
%          cmd 命令行:texdoc thubeamer 查看文档
%          推荐前往模板的 GitHub 仓库获取最新文件,地址:
%          https://github.com/YangLaTeX/thubeamer
%%=======================================================================

% 设置文档类别为 <beamer>
% \documentclass[aspectratio=169]{beamer} % 设置长宽比为 16:9
\documentclass{beamer}
\usepackage{url}
% 使用 <thubeamer> 主题
% 模板选项如下
% (a.1) smoothbars: 页面顶端单行显示目录,默认选项
\usetheme{thubeamer}
% \usetheme[sidebar]{subsectiontoc}
% (a.2) sidebar: 页面左侧分栏显示目录
% \usetheme[sidebar]{thubeamer}

% (b) sectiontoc: 在每节(section)前显示目录,并高亮显示当前节,默认不显示

% (c) subsectiontoc: 在每小节(subsection)前显示目录,并高亮显示当前节和当前小节,默认不显示

% (d) en: 仅使用英文来制作 beamer,应用此选项后,汉字将全部无法编译

% 图片存放路径
\graphicspath{{figures/}}

\include{macros}

% 封面信息,方括号内容是显示在左侧边栏的内容(当选择 sidebar 主题时有效)
\title[共享调度器技术报告]{共享调度器技术报告}
\author[赵方亮、廖东海]{项目成员:赵方亮、廖东海\\[5mm] 导师:向勇 教授}
\institute[清华大学]{\small  清华大学}
\date{\small \vskip -10pt \today}

% 开始写文章
\begin{document}

% 标题页
\begin{frame}
	\maketitle
\end{frame}

% 目录页
\section*{目录}
\frame{
  \frametitle{\secname}
  \tableofcontents[hideallsubsections]
}

\section{项目背景}

\subsection{项目基础}

\begin{frame}{项目基础}
  \begin{block}{用户态中断\cite{gallium70}}
    \begin{itemize}
      \item 外部中断
      \item 内核向用户进程发送中断
    \end{itemize}
  \end{block}
  \begin{block}{Rust 协程机制\cite{stevenbai}}
    \begin{itemize}
      \item async/await
      \item Executor 运行时
    \end{itemize}
  \end{block}
\end{frame}

\subsection{项目整体概况}

\begin{frame}{项目整体概况}
  \begin{figure}
    \includegraphics[width=0.9\linewidth]{figures/overview.png}
    \caption{项目整体概况}
  \end{figure}
\end{frame}


\section{研究工作}

\begin{frame}{已完成的研究工作及成果}
  \begin{block}{已完成的研究工作简介}
    \begin{itemize}
      \setlength{\itemsep}{6pt}
      \item 任务调度
      \item 系统调用
      \item 内核改造
      \item vDSO 机制
      \item 协程调度框架
    \end{itemize}
  \end{block}
\end{frame}

\subsection{任务调度}

\begin{frame}{任务调度}
  \begin{block}{进程、线程调度}
    \begin{itemize}
      \setlength{\itemsep}{6pt}
      \item 由内核中的调度协程进行
      \item 优先级:依赖协程优先级
      \item 线程:平等
    \end{itemize}
  \end{block}
  \begin{block}{协程调度}
    \begin{itemize}
      \setlength{\itemsep}{6pt}
      \item 用户态:共享调度器
      \item 优先级队列
    \end{itemize}
  \end{block}
\end{frame}

\begin{frame}{状态转换模型}
  \begin{figure}[htbp]
    \centering
    \includegraphics[width=1.0\linewidth]{figures/tsm.png}
    \caption{状态转换模型}
    \label{fig:tsm}
  \end{figure}
\end{frame}

\subsection{系统调用}

\begin{frame}{系统调用(以 read 系统调用为例)}
  \begin{block}{接口}
    同步与异步的接口进行统一,通过参数进行区别
    \begin{itemize}
      \item {read!(fd, buffer);}
      \item {read!(fd, buffer, key, cid);}
    \end{itemize}
  \end{block}
  \begin{block}{功能扩展}
    定义 AsyncCall 数据结构,实现 Future 特性,让系统调用阻塞在用户态
  \end{block}
\end{frame}

\subsection{内核改造}

\begin{frame}{内核改造}
  \begin{block}{内核调度协程}
    \begin{itemize}
      \item 主动让权
    \end{itemize}
  \end{block}
  \begin{block}{系统调用处理}
    以 read 系统调用为例 
    \begin{itemize}
      \item 创建内核协程:立即返回
      \item 唤醒用户协程:内核协程完成复制操作,发起用户态中断唤醒用户态协程
    \end{itemize}
  \end{block}
\end{frame}

\subsection{vDSO 机制}

\begin{frame}{vDSO 机制}
  共享调度器由内核进行维护,以 vDSO 的形式暴露给用户进程
  \begin{block}{加载与动态链接}
    \begin{itemize}
      \item 创建进程时映射共享调度器
      \item 查找进程符号表,进行链接
    \end{itemize}
  \end{block}
  \begin{block}{堆、Executor}
    \begin{itemize}
      \item 全局堆分配器
      \item 全局 Executor
    \end{itemize}
  \end{block}
  
\end{frame}

\subsection{协程调度框架}

\begin{frame}{协程调度框架}
  协程调度框架依托于 Executor 数据结构,为其实现不同的成员函数,从而实现不同的调度算法
  \begin{figure}[htbp]
    \centering
    \includegraphics[width=0.8\linewidth]{figures/sf.png}
    \caption{协程调度框架}
    \label{fig:sf}
  \end{figure}
\end{frame}


\section{性能评估}

\subsection{性能评估}

\begin{frame}{线程、协程对比}
  \begin{figure}[htbp]
    \centering
    \includegraphics[width=1.0\linewidth]{figures/mdmc.png}
    \caption{线程、协程时延对比}
    \label{fig:mdmc}
  \end{figure}
  \begin{itemize}
    \setlength{\itemsep}{6pt}
    \item 连接数较少时,线程模型与协程模型持平,随着连接数增加,线程模型时延迅速上升,远高于协程
    \item 数据规模增大,线程模型时延增加幅度远大于协程模型
  \end{itemize} 
\end{frame}

\begin{frame}{线程、协程对比}
  \begin{figure}[htbp]
    \centering
    \includegraphics[width=0.6\linewidth]{figures/tmc.png}
    \caption{线程、协程吞吐量对比}
    \label{fig:tmc}
  \end{figure} 
  \begin{itemize}
    \setlength{\itemsep}{6pt}
    \item 连接数增加,线程模型吞吐量增长缓慢
  \end{itemize} 
\end{frame}

\subsection{优先级}

\begin{frame}{优先级}
  \begin{figure}[htbp]
    \centering
    \includegraphics[width=1.0\linewidth]{figures/cwp.png}
    \caption{不同优先级下的时延与抖动}
    \label{fig:cwp}
  \end{figure}
  \begin{itemize}
    \setlength{\itemsep}{6pt}
    \item 随着 CPU 资源数量的增加,由于调度器同步互斥带来的开销,高优先级的连接出现小幅度的性能下降
  \end{itemize}
\end{frame}

\begin{frame}{优先级}
  \begin{figure}[htbp]
    \centering
    \includegraphics[width=0.6\linewidth]{figures/cwpdc.png}
    \caption{不同优先级下的时延分布}
    \label{fig:cwpdc}
  \end{figure}  
  \begin{itemize}
    \setlength{\itemsep}{6pt}
    \item 资源有限时,保证了优先级高的任务
  \end{itemize}
\end{frame}

\section{未来展望}

\begin{frame}{未来展望}
  \begin{block}{下一步工作}
    \begin{itemize}
      \item 文件系统扩展
      \item 进程、线程、协程调度
    \end{itemize}
  \end{block}
\end{frame}


\begin{frame}[allowframebreaks]{参考文献}
  \bibliographystyle{thubeamer}
  \bibliography{reference}
\end{frame}

\begin{frame}
	\begin{center}
    {\Huge\calligra Thanks for your attention!}
    \vspace{1cm}

    {\Huge Q \& A}
  \end{center}
\end{frame}

% 结束文档撰写
\end{document}
