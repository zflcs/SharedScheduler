\section{协程调度框架}

内核和所有的用户进程能够访问 SharedScheduler 暴露的共享库。但这些操作最终会落实到对其内部的 Executor 数据结构的操作上。为了避免在内核和用户库中重复定义 Executor 数据结构,我们在独立的 lib 中定义 Executor,并为其实现某些成员方法。

按照 Rust 的特性,仅仅在内核和用户进程中创建 Executor 数据结构而不使用其成员方法,那么内核和用户程序编译生成的 ELF 文件中将不会包含这些成员方法。而在 SharedScheduler 中,我们通过调用 Executor 的成员方法完成对它内部的操作。这样一种特性,使得我们既能保留住共享的特点,又能方便的实现模块化的协程调度框架,从而通过不同的 feature 使用不同的协程调度算法,见图 4。

