\section{功能评价}

我们的项目脱胎于AmoyCherry的[AsyncOS](https://github.com/AmoyCherry/AsyncOS)开源项目,采用了相同的优先级的调度算法,但在调度器的内核与用户态复用、异步协程唤醒、内存分配方面有了不同程度的改进,同时也扩展了调度器对多核多线程的支持。为了衡量这些因素对性能的影响,我们设置了串行的pipe环实验与 AsyncOS 进行性能对比,同时展示多线程的加速情况。

此外,为了展示协程相比于线程所拥有的低时延的切换开销,我们利用协程和线程搭建了两个不同模型的WebServer来测试服务器的吞吐量、消息时延和抖动的情况,由于当前的实验框架暂不支持socket网络通信,我们使用Pipe通信来模拟网络中的TCP长连接。

最后,我们还将通过WebServer的实验展示协程优先级对任务的吞吐量、消息时延和抖动的影响,展示优先级在有限资源下保障某些特定任务实时性上的巨大作用。

总体实验环境:[qemu 5.0 for Riscv with N Extension](https://github.com/duskmoon314/qemu)、4个CPU核心。

