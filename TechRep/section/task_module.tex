\section{任务调度与状态转换模型}

\subsection{任务调度}

我们保留进程控制块(PCB)、线程控制块(TCB)等数据结构不变,进程仍然负责地址空间隔离,而线程不再与特定的某个任务(函数)进行绑定,而是被视为分配给某个进程的虚拟 CPU,负责运行协程的调度函数,提供给协程代码运行所需要的栈。

同时,使用协程控制块(CCB)来描述任务信息。图 2 展示了内核和用户进程地址空间内的数据结构。其中,PCB 和 TCB 处于内核地址空间中,由内核进行管理;CCB 处于内核和用户进程地址空间内的 Executor 数据结构中,由内核和用户进程各自对其进行管理。由于这些数据结构所处的地址空间不同,而要对任务进行调度,因此整个系统的运行需要进行两种调度:(1)由内核完成进程、线程调度;(2)内核、用户进程内部的协程调度。图 2 展示了这两种调度。


一般来说,进程、线程都需要进行调度,需要根据一定的策略,将 CPU 分配给处于就绪队列中的某个进程中的某个线程。在没有引入协程之前,线程绑定了特定的任务(函数),这个任务同样存在优先级,因此在进程调度完成之后还需要进行一次线程调度。

但是,基于上述对线程概念变化的描述,线程并不与某个具体的任务进行绑定,TCB 中不需要 prio 字段记录任务的优先级,所有线程都是等价的。因此,这一阶段不需要额外进行调度。我们将这两种调度合并在一起,由内核来完成,从所有线程的就绪队列中找到属于目标进程的位置最靠前的线程即可,如图 2 所示。

总体来说,进程、线程调度仍然需要以优先级作为依据,在进程初始化时,我们将进程的优先级设置为默认优先级,一旦进程开始运行,其优先级会动态变化,这种变化将会在协程调度中进行介绍。


利用 Rust 语言提供的 async/await 关键字,我们可以轻易的利用协程来描述任务,同时也是由于 Rust 语言的机制,协程必须依靠异步运行时(AsyncRuntime)才能运行。为此,我们以共享库的形式将 Async Runtime 的依赖提供给内核和用户进程,即共享调度器(SharedScheduler)。

如图 2 所示,协程作为任务单元,每个协程都应该有各自的优先级,保存在协程控制块的 prio 字段中,SharedScheduler 采用了优先级位图来进行调度,位图中的每位 bit 对应着一个优先级队列,1 表示这个优先级的队列中存在就绪协程,0 表示这个优先级队列内不存在就绪协程。进行协程调度时,根据优先级位图,取出优先级最高的协程并更新位图,使其与进程内的协程的优先级始终保持一致。

上文提到的进程优先级动态变化与协程的调度相关。目前的实现是将进程优先级的实际意义理解为进程内协程的最高优先级,它记录在 SharedScheduler 内提供了一段共享内存中,实际上是优先级数组。在完成对内核或者用户进程内的协程的一次调度之后,SharedScheduler 会在优先级数组上对应的位置更新进程内协程的最高优先级。

\subsection{状态转换模型}

引入协程之后,任务的调度发生了变化,因此,相应的状态转换模型也产生了一系列变化。由于内核不感知协程,只负责进程、线程调度,线程的状态模型仍然是成立的。除此之外,我们还为协程建立了和线程类似的状态模型。由于协程调度建立在线程调度完成的基础上,因此协程的状态转换依附于线程的状态转换,却又会导致线程的状态发生变化,二者相互影响。如图 3 所示。


当线程处于就绪和阻塞的状态时,其内部协程的状态不会发生变化。只有在线程处于运行状态,协程才会发生状态转换。通常,线程会由于内核执行了某些处理过程,从阻塞态恢复到就绪态时,线程内部的协程所等待的事件也已经处理完毕,此时协程理应处于就绪的状态,因此,需要用某种方式将协程的状态从阻塞转换为就绪态。我们通过用户态中断完成了这个过程,但是,用户态中断唤醒协程目前的实现实际上是新开了一个线程专门执行唤醒的操作,因此,协程状态发生变化还是在线程处于运行的状态。这里存在着二者相互影响的逻辑关系。

协程在创建之后进入就绪态,经过调度转变为运行态;在运行时,当前协程可能会因为等待某一事件而进入到阻塞态,也可能因为检测到存在其他优先级而让权,也可能会执行完毕进入退出状态;当协程处于阻塞的状态时,这时只能等待某个事件发生之后被唤醒从而进入就绪的状态。


在图 3 中,不仅仅描述了线程内部的协程状态模型,同时还描述了线程状态模型以及线程、协程状态转换的相互影响关系。从图中我们可以观察到,线程仍然具有创建、就绪、阻塞、运行和结束五种状态,其状态转换与上述 3.2.1 节类似。需要注意的是,尽管从内核的视角看来,造成线程发生状态转移的事件仍然是调度、让权、等待以及唤醒这几类。但是,线程内部的协程状态转换给这几类事件增加了新的起因。因此,造成线程状态变化的原因具体可划分为以下几类:
(1)与内部协程无关的外部事件。例如,无论内部的协程处于何种状态,都会因为时钟中断被抢占进入就绪态。
(2)与内部协程相关的外部事件。由于内核执行了某些处理过程,完成了线程内部协程所等待的事件,使得协程需要被重新调度,线程将会从阻塞态转换为就绪态。
(3)内部协程自身状态变化。一方面是协程主动让权,当上一个执行完毕的协程处于就绪或阻塞态,需要调度下一个协程时,若此时检测出其他进程内存在更高优先级的协程,这时会导致线程让权进入就绪态;另一方面是协程执行过程中产生了异常,导致线程进入阻塞态,等待内核处理异常。

然而,无论何种原因,一旦线程状态发生变化,就会发生线程切换。这种切换需要保存 CPU 上的所有寄存器信息,是比较低效的。
