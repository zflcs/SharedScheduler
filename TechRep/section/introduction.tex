\section{背景介绍}

\subsection{用户态中断}

在传统的操作系统中,我们常常使用内核转发的方式模拟信号传输,来进行进程间通信。发送信号的进程需要进入内核态进行操作,特权级的切换造成的性能开销将导致效率低下,而接收信号的进程往往需要等待下一次被调度时才能响应中断,这意味这信号不能被及时响应,导致延迟较高。用户态中断技术的出现,使得高效的信号传输成为可能。

“中断”的术语在之前常常用于描述源自硬件并在内核中处理的信号。即使是我们常用的软件中断(即软件产生的信号)也是在内核中处理的。但近几年出现的一种设计,希望处于用户空间的各个任务可以相互之间不通过内核,直接发送中断信号并在用户态处理。2020年,Intel在Sapphire Rapids处理器中首次发布x86的用户态中断功能,而2021年的Linux 在RFC Patch中提交了对用户态中断支持的代码,我们以用户态的信号传输为例,对用户态中断的优势进行分析。

在传统的多核系统中,运行在两个核心的用户进程通过信号进行通信时,发送端进程需要陷入内核,向接收端进程的信号队列中插入信号,等待接受进程下一次被调度时才能响应信号;而在支持用户中断的多核系统中,发送端可以直接通过硬件接口向接收端发送信号,而接收端在核上调度时会立即响应。

\subsection{Rust 语言协程机制}
